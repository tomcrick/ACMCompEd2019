% !TEX TS-program = pdflatex
% !TEX encoding = UTF-8 Unicode

% This is a simple template for a LaTeX document using the "article" class.
% See "book", "report", "letter" for other types of document.

\documentclass[11pt]{article} % use larger type; default would be 10pt

\usepackage[utf8]{inputenc} % set input encoding (not needed with XeLaTeX)

%%% Examples of Article customizations
% These packages are optional, depending whether you want the features they provide.
% See the LaTeX Companion or other references for full information.

%%% PAGE DIMENSIONS
\usepackage{geometry} % to change the page dimensions
\geometry{a4paper} % or letterpaper (US) or a5paper or....
% \geometry{margin=2in} % for example, change the margins to 2 inches all round
% \geometry{landscape} % set up the page for landscape
%   read geometry.pdf for detailed page layout information

\usepackage{graphicx} % support the \includegraphics command and options

% \usepackage[parfill]{parskip} % Activate to begin paragraphs with an empty line rather than an indent

%%% PACKAGES
\usepackage{booktabs} % for much better looking tables
\usepackage{array} % for better arrays (eg matrices) in maths
\usepackage{paralist} % very flexible & customisable lists (eg. enumerate/itemize, etc.)
\usepackage{verbatim} % adds environment for commenting out blocks of text & for better verbatim
\usepackage{subfig} % make it possible to include more than one captioned figure/table in a single float
% These packages are all incorporated in the memoir class to one degree or another...
\usepackage{url}
%%% HEADERS & FOOTERS
\usepackage{fancyhdr} % This should be set AFTER setting up the page geometry
\pagestyle{fancy} % options: empty , plain , fancy
\renewcommand{\headrulewidth}{0pt} % customise the layout...
\lhead{}\chead{}\rhead{}
\lfoot{}\cfoot{\thepage}\rfoot{}

%%% SECTION TITLE APPEARANCE
\usepackage{sectsty}
\allsectionsfont{\sffamily\mdseries\upshape} % (See the fntguide.pdf for font help)
% (This matches ConTeXt defaults)

%%% ToC (table of contents) APPEARANCE
\usepackage[nottoc,notlof,notlot]{tocbibind} % Put the bibliography in the ToC
\usepackage[titles,subfigure]{tocloft} % Alter the style of the Table of Contents
\renewcommand{\cftsecfont}{\rmfamily\mdseries\upshape}
\renewcommand{\cftsecpagefont}{\rmfamily\mdseries\upshape} % No bold!

%%% END Article customizations

%%% The "real" document content comes below...

\title{Brief Article}
\author{The Author}
%\date{} % Activate to display a given date or no date (if empty),
         % otherwise the current date is printed 

\begin{document}
%\maketitle
\leftline{\bf Poster Proposal for SIGCSE 2019}

\leftline{\bf Proposers:}
\leftline{James H.~Davenport, University of Bath, U.K., {\tt J.H.Davenport@bath.ac.uk}, \url{http://staff.bathac.uk/masjhd}}
\leftline{\bf On behalf of the Institute of Coding Consortium}
 
\leftline {\bf Title:}\noindent
The Institute of Coding: A University-Industry Collaboration to Address the UK Digital Skills Crisis

\leftline{\bf Abstract:}\noindent
The UK is not the only country with a serious digital skills crisis, but it is one with a formal Government inquiry (The Shadbolt Report) and response.
The Institute of Coding is a new \pounds40m+ initiative by the UK
Government to transform the digital skills profile of the country. It responds to the apparently contradictory data that
the country has a digital skills shortage across a variety of sectors,
yet unemployed computing graduates every year. The Institute is a
large-scale national intervention to address some of the perceived
issues with formal education versus industry skills and training, for
example: technical skills versus soft skills, industry-readiness
versus ``deep education'', and managing expectations for the diverse
digital, data and computational skills demands of employers across a
wide range of economic sectors.

All of this is taking place in the higher education/workforce domain
at the same time as substantial levels of computer science curriculum
reform across the four nations of the UK -- especially in England,
with a new computing curriculum that first started in 2014, in which
all children are expected to learn two programming languages, as well
as wider computer science fundamentals and computational thinking
skills.

In this poster, we describe the background and evidence base for the
Institute of Coding, its key themes and current activities, as
well as reflecting on potential replicability of aspects of the
Institute to others with similar ambitions.

\medskip
\leftline{\bf Significance and Relevance of the Topic:}\noindent
We do not know of an educational system that does not worry about the ``digital skills crisis'' in some form or other. There are innumerable reports, and many Government or grass-roots (such as CS4ALL) initiatives.  These tend to be addressed at the school sector, with the (generally unspoken) premise being that, if only more schoolchildren were interested in this area, more would study it at university and all would be well in the (alas distant) future, as Higher Education will do what more is necessary.

However, the UK Government, with its extremely detailed tracking of individuals through, and out of, the Higher Education, has observed that, at least superficially, all is not well with the Higher Education sector in the UK when it comes to digital skills. The headline issue was that Computer Science graduates had a higher rate of unemployment: 11.7\%  six months after graduation, compared with a STEM average of 8.4\%. Hence the Shadbolt enquiry was set up. It rapidly fund that reality is more complicated, with Computer Science reaching much deeper into the socio-economic ladder than most other STEM subjects.  Another concern identified was the wide inter-university differences in unemployment rates, inexplicable on the basis of prior achievement.

Largely in rsponse to Shadbolt, the UG Goverment proposed a national Institute of Coding in 2015, which by the time the formal bid emerged in 2017 has become an England-only initiative aimed at Higher Education and similar levels of Education.  The Institute was formally launched in 2018, with \pounds20M of Goverment money and at least as much in matching funding.

\medskip
\leftline{\bf Content:}\noindent
The poster will be in two columns, broadly speaking as follows.


\medskip
\leftline{\bf Left:}\noindent
This column will describe the UK situation, and the evidence of a mismatch in UK Higher Education.  In particular we will note some more recent gender data, whch indicates widely differing outcomes (in terms of median salary) between male and female graduates from some, but far from all, institutions.


\medskip
\leftline{\bf Right:}\noindent
This column will describe the Institute of Coding's structure, the various partners, both university and industry involved, and how we are achieving genuine industry input into the education being delivered.

We will outline the various themes and work packages into which it is split, and sketch the progress being made so far.

\medskip
\leftline{\bf Support}\noindent
By the time of SIGCSE 2019, the Instiute of Coding will have launched its professional website, which will contain much backing information. We will have fliers and various supporting papers available, but largely expect the poster to generate interest in the website and the information there.

\leftline{\bf James Davenport}\noindent
He has taught Computer Science and Mathematics at the University of Bath for over 35 years, in which time he has, for example, taught MATLAB to one British citizen in 25,000. The University of Bath is the lead insitution in the Institute of Coding.
\end{document}
