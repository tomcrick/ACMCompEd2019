%%%% Proceedings format for most of ACM conferences (with the exceptions listed below) and all ICPS volumes.
\documentclass[sigconf]{acmart}
\usepackage{paralist}
\usepackage{url}
\usepackage[hyphenbreaks]{breakurl}

\def\UrlBreaks{\do\/\do-}

%%%% As of March 2017, [siggraph] is no longer used. Please use sigconf (above) for SIGGRAPH conferences.

%%%% Proceedings format for SIGPLAN conferences 
% \documentclass[sigplan, anonymous, review]{acmart}

%%%% Proceedings format for SIGCHI conferences
% \documentclass[sigchi, review]{acmart}

\usepackage{booktabs} % For formal tables


% Copyright
%\setcopyright{none}
%\setcopyright{acmcopyright}
\setcopyright{acmlicensed}
%\setcopyright{rightsretained}
%\setcopyright{usgov}
%\setcopyright{usgovmixed}
%\setcopyright{cagov}
%\setcopyright{cagovmixed}

\copyrightyear{2019}
\acmYear{2019}
\setcopyright{acmlicensed}
\acmConference[SIGCSE '19]{The 50th ACM Technical Symposium on
  Computer Science Education}{Feb. 27--Mar 2, 2019}{Minneapolis, MN, USA}
%\acmBooktitle{}
%\acmPrice{15.00}
%\acmDOI{10.1145/3159450.3159547}
%\acmISBN{978-1-4503-5103-4/18/02}
% This slight change to the code may also save 1 or 2 lines of space.

% removes the headers from each page per the preparation instructions, as these are not needed and will be updated with the chairs' actual session names during the pagination/indexing process:
\fancyhead{}

\begin{document}
\title{The Institute of Coding: A University-Industry Collaboration to Address the UK Digital Skills Crisis}
%\titlenote{}
%\subtitle{Extended Abstract}
%\subtitlenote{}

\author{James H. Davenport}
\orcid{0000-0002-3982-7545}
\affiliation{%
  \institution{University of Bath}
  \streetaddress{}
  \city{Bath} 
  \country{United Kingdom}
}
\email{j.h.davenport@bath.ac.uk}

\author{Tom Crick}
\orcid{0000-0001-5196-9389}
\affiliation{%
  \institution{Swansea University}
  \streetaddress{}
  \city{Swansea} 
  \country{United Kingdom}
}
\email{thomas.crick@swansea.ac.uk}

\author{Alan Hayes}
%\orcid{}
\affiliation{%
  \institution{University of Bath}
  \streetaddress{}
  \city{Bath} 
  \country{United Kingdom}
}
\email{a.hayes@bath.ac.uk}

\author{Rachid Hourizi}
%\orcid{}
\affiliation{%
  \institution{University of Bath}
  \streetaddress{}
  \city{Bath} 
  \country{United Kingdom}
}
\email{r.hourizi@bath.ac.uk}

 
% The default list of authors is too long for headers}
\renewcommand{\shortauthors}{Davenport, Crick, Hayes and Hourizi}


\begin{abstract}
The Institute of Coding is a new c.\pounds40m initiative by the UK
Government to transform the digital skills profile of the country. In
the context of significant national and international political and
policy scrutiny, it responds to the apparently contradictory data that
the country has a digital skills shortage across a variety of sectors,
yet unemployed computing graduates every year. The Institute is a
large-scale national intervention to address some of the perceived
issues with formal education versus industry skills and training, for
example: technical skills versus soft skills, industry-readiness
versus ``deep education'', and managing expectations for the diverse
digital, data and computational skills demands of employers across a
wide range of economic sectors.

All of this is taking place in the higher education/workforce domain
at the same time as radical levels of computer science curriculum
reform across the four nations of the UK -- especially in England,
with a new computing curriculum that first started in 2014, in which
all children are expected to learn two programming languages, as well
as wider computer science fundamentals and computational thinking
skills.

In this paper, we describe the background and evidence base for this
new national initiative, its hybrid structure, key themes and planned
deliverables, identifying some of the key opportunities and
challenges, as well as its potential replicability of aspects of the
Institute in other jurisdictions.
\end{abstract}

\keywords{Computer science education, digital skills, programming, higher
  education, industry relevance, UK}

\maketitle

% From CfP:
% Short papers (up to 5 pages) focus on dissemination and discussion
% of new ideas in computing education practice or research that merit
% wider awareness and discussion within the community. They can present
% preliminary results of new educational innovations, present and
% discuss novel educational technologies, report work-in-progress
% research (including promising systems or tools that have not yet been
% evaluated and/or adopted extensively), or raise issues of significance
% for the development of the discipline, such as long-term strategic
% needs for computing education and curricula. All short papers are
% expected to have an appropriate coverage of literature to support the
% ideas and arguments that they present. Because it lacks some elements
% of a research paper, a short paper is evaluated mainly by its
% anticipated impact on discussions during the conference and possible
% future contribution to the field of computing education.

\section{Introduction}

% TC: need more UK context here, I can add this -- do we want to have
% this framed as a paper to justify/introduce the IoC, as well as
% other UK interventions? Framed as a discussion piece with
% transferability elsewhere?

Superficially, the employment outlook for computing graduates in the
UK looks excellent. \cite[p.~74]{UKCES2015b} states
\begin{quote} the digital sub-sector will need 518,000 workers for
roles in the three highest skilled occupational groups. However, over
the last ten years only 164,000 individuals graduated from a first
degree in computer science.
\end{quote} This is profitable for the individual: according to
\cite[Figure 4]{BIS2011a}, ``mathematical and computer sciences'' have
the second highest earnings return of all subjects (beaten only by
``medicine and dentistry'').  The country profits from this as well:
according to \cite[p.~16]{BIS2011a}, this is, per head, the fourth
most beneficial subject to the Exchequer.

% TC: add note about nomenclature/terminology -- not hugely happy with
% "Institute of Coding'', but this was political, etc...

\section{So What's the problem}

Despite the headline success in \cite{BIS2011a}, the employment
figures are not great, and the earnings data are patchy.

\subsection{Employment}

Quoting \cite{UKCES2015b}, the author of a UK Government-commissioned
report \cite{Shadbolt2016a} writes

\begin{quote} In this context, apparently high rates of
unemployment\footnote{11.7\% six months after graduation (the standard
UK measure) at the time of \cite{Shadbolt2016a}, compared with a STEM
average of 8.4\%. Note, however, that Computing is 20\% of STEM
\cite[Table 1]{Wakeham2016a}, so `STEM-less-Computing' has a 7.6\%
unemployment rate.} amongst graduates of Computer Sciences and other
STEM\footnote{STEM is ``Science, Technology, Engineering,
Mathematics'' for \cite{Shadbolt2016a} and this paper.} courses
demanded an explanation.
\end{quote}

A significant explanation is ``There are notable differences in the
characteristics of Computer Sciences entrants compared to entrants in
other STEM subjects'' \cite[\P2.6]{Shadbolt2016a}: fewer women, but

\begin{description}
\item[50\% more] mature students,
\item[16\% more]Black and Minority Ethnic (BME) and
\item[40\% more]students from backgrounds where people have
traditionally not participated in HE (LPNs).
\end{description}

Mature, BME and LPN students all find getting jobs more difficult.
\par However, for those students that do find jobs, the data are
better, showing \cite[Figure 6]{Shadbolt2016a} fewer students in
``non-graduate jobs'' or low-earning jobs than in STEM as a whole.

\subsection{Earnings}
If we look beyond purely getting jobs to the earnings\footnote{Clearly
not the only measure of job quality, or contribution to society, but
at least it's measurable, and has been measured in the LEO dataset
\cite{DfE2017a}, which tracks individuals through school, university
and into the labour market, combining educational, tax and benefits
data.}, the position (as described in \cite{DfE2018d}, and presented
to the public in \cite{BBC2018f}, which also allows the reader to
break down the data by university and subject.) is even less clear on
a microscopic level, though on a macroscopic level it bears out much
of what \cite{Shadbolt2016a} said.

On the macroscopic level, the reader should consider \cite[Table
5]{DfE2018d}. We focus on the `Men' data as presented here, as there
are (regrettably) many more than there are women in the cohorts,
though the effects are similar. This shows that an OLS (``Ordinary
Least Squares'') fit shown that a man reading Computing would earn
3.3\% more than had he read a subject at random. If one corrects for
prior attainment, this rises to 10.5\%, and 12.6\% if other factors
are taken into account. For reasons explained in
\cite[\S4.2]{DfE2018d}, the authors prefer IPRWA (``Inverse
Probability Weighted Regression Adjustment''), and this moves the
earning difference to 14.4\%. For men, the overall effect of these
adjustments is to move Computing from being middle-of-the-pack
\cite[Figure 15]{DfE2018d} to fourth best \cite[Figure 17]{DfE2018d},
and for women it moves to seventh best \cite[Figure
16]{DfE2018d}. Note that these are improvements on the average
graduate earnings which are \pounds30,000/year for men and
\pounds26,000/year for women \cite[p. 37]{DfE2018d}. Hence if a
particular subject were sending students into a gender-neutral world,
the women would be showing a 15\% (\pounds4,000/year) premium just to
catch up with the men.

\subsection{Per-University Earnings}

\cite{BBC2018f} allows one to break down the data underpinning
\cite{DfE2018d}, and the Computing figures are challenging.  Salary
premiums, allowing for the factors described above, are reported
separately for men and women, and only if there were at least 50
students of that gender in the five cohorts (graduation 2007--8 to
graduation 2011-12) considered. This means that, of the 82 English
universities reporting computing, 80 report male data and 30 report
female data --- 28 report both. Looking at the 28 (see Figure
\ref{fig:BBC}), one's first impression is that the male and female
data are uncorrelated: for example the two universities with male
premiums just above +\pounds2500 have female premiums of +\pounds9325
and -\pounds5793. There is in fact a definite ($p=0.0034$) positive
correlation, but a fairly weak one ($R^2=0.286$). The best fit is
$W=0.92672+0.53388*M$. For the reasons explained at the end of the
previous section, the ideal ``gender-neural" fit would be
$W=4+M$. Both these lines are shown in Figure \ref{fig:BBC}.

\begin{figure}\caption{\label{fig:BBC}} \hbox{\hskip-2em
\includegraphics[scale=0.34]{BBCSalaryDatav4.jpg}}
\end{figure}

\section{UK Policy Context}

It is worth recalling that, after the Government's acceptance of the
Browne report \cite{BIS2010a}, students in England pay probably the
highest\footnote{Or possibly second-highest after US students, but the
US averages in \cite[Table B5.1]{OECD2016a} conceal an enormous
variation.} prices in the world for undergraduate education: between
\pounds6000 and \pounds9000/year for tuition alone. While this is
normally covered by student loans repaid on an income-contingent
basis, essentially through a 9\% income tax premium, there is evidence
\cite{CallenderMason2017a} that this ``contribute[s] to lower rates of
planned H[igher] E[ducation] participation by lower-class students''.

\subsection{Teaching Excellence Framework}
AH to write overview paragraph. But JHD will try.

UK universities have been judged, very publically, on their research for the last thirty years by the Research Assessment Exercise, and its successor the Research Evaluation Framework (REF). This has led to many complaints, not entirely unjustified, that teaching, because it is not measured, is not taken as seriously. Similar comments in the USA can be found in \cite{Campbelletal2018a}. To counteract this, the Government introduced a `` Teaching Excellence Framework (TEF)''.
The initial versionb of the TEF produces a university-wide assessment on a three-pont (Gold/Silver/Bronze) scale. It in fact used no assessment of teaching as such, merely statistical data such as that used in \cite{Shadbolt2016a}.

In 2018, following an Institutional-level review undertaken in the previous year, the Office for Students (OFS) conducted a pilot of a national  subject-based TEF. The aim of the TEF was to evaluate the quality of HE provision for each Higher Education provider within the United Kingdom. The objective was to provide sufficient data for prospective students to enable them to undertake an informed decision about their choice of University and subject of study. This was particularly pertinent to the UK context as the funding of Higher Education has shifted, following severla policy changes, the latest being the
Browne report \cite{BIS2010a}, from students receiving means-tested Government grants to undertaking a loan to pay for their tuition. % Typical tuition fees are set at around £9k per annum.

Fifty higher education providers participated in the 2018 pilot, reflecting the diverse range of UK HE within the sector. The pilot collated subjects into 7 groups with 142 subject panel members reviewing the provision. The subject panels consisted of members of the academic community, students, employers and representatives of the professional bodies. Reviewing panels rated a HE provider's subject provision as being either Gold, Silver or Bronze. The Computing subject was grouped with Engineering and Technology. The main goal of the pilot was to inform the development of the subject-TEF methodology to be adopted by the national roll-out currently scheduled for 2019/20. The framework evaluated subject-provision against 6 core metrics. Three of these came directly from the National Student Survey (teaching, academic support and assessment and feedback) one for continuation (retention) and two were related to employment (employment or further study/ highly skilled employment or further study). Reviewers are able to review a provider's metrics against a range of stakeholders including full/part-time students, ethnic diversity, gender, age profile etc. Additionally, supplementary statistics are provided to inform the panel member's review. These include teaching intensity (contact hours in all its various guises) and long-term employment/earnings as measured by the Longitudinal Education Outcomes (LEO) data.

One consequence of the  TEF metrics is that there is a proposed strong correlation between the quality of provision and the employment prospects/performance of the students. This can be challenging for some disciplines (e.g. income earning potential is not equally distributed across all subjects). The intention however is clear. In the context of students paying their own tuition fees, quality of provision is being linked to future employment prospects. This represents an opportunity for HE Computing provision. Those programmes, characteristic of most in computer science,  that contain a placement/internship, are professional accredited and whose curriculum is informed by an employer-led advisory board are well-placed to do well in the TEF. 



\subsection{Degree Apprenticeships}

The Government has also launched ``Degree Apprenticeships''
\cite{BIS2015a}. These were described by the then Prime Minister as
``combining a full degree with the real practical skills gained in
work and the financial security of a regular pay packet''. The
employer pays the tuition cost, but due to the Apprenticeship Levy
\cite{HMRC2016a}, most employers will find there is no net cost.

Set the scene -- UK digital skills, CS ed reform, digital economy, etc

Cite previous work on
programming~\cite{davenport-et-al:latice2016,murphy-et-al:programming2017,simon-et-al:sigcse2018}

Cite previous policy-related
work~\cite{crick+sentance:2011,brown-et-al:sigcse2013,brown-et-al:toce2014,crick+moller:wipsce2015,moller+crick:jce2018}

\section{Skills mismatch}

There is a widespread and longstanding complaint that ``students
aren't industry-ready'', or ``there is a skills mismatch''. Some of
this is due to a misapprehension on the part of employers (generally
those outside the IT industry itself, but seeking to hire people wih
``10 years experience of programming in Ruby''), but much of it is
genuine. One of the main challenges for the university community is to
understand this complaint.

\subsection{Sandwich Years}

In the U.K. context, a university course that includes a period
working in industry (which may include government, charity etc.) is
generally called ``sandwich'', and in North America the term ``co-op''
is generally used. The most common model in Computing in England,
where the vast majority of students study three-year Bachelor's
degrees, is a year's placement in industry between the second and
final years of study. This is remarkably successful in computing. The
University of Bath has run such courses since its founding (1966),
with about 80\% of students opting to take the sandwich year. There is
statistical evidence for its success wherever it is used in the
U.K.\footnote{And at least anecdotal evidence elsewhere: ``the co-op
system is a major reason for our [University of Waterloo] success''
\cite{Watt2017a}.}:

\begin{quote} those studying sandwich courses enjoy the lowest levels
of unemployment (6\% sandwich vs 15\% non-sandwich), the lowest levels
of non-graduate level employment (6\% sandwich vs 25\% non-sandwich),
and graduates from sandwich courses are twice as likely to be earning
over \pounds20,000 compared to those who did a standard
degree. \cite[\P2.5]{Shadbolt2016a}
\end{quote}

A simplistic remedy would be to require that all students study
sandwich degrees, but this has numerous objections:

\begin{enumerate}
\item Some students do not wish to, often for valid reasons;
\item The supply of employers willing to offer such placements is
limited, and often they are only offered to a limited nmber of
universities with whom the employer has built up relations, often
going back decades;
\item The university needs to invest in the process: a successful
sandwich year programme is not a matter of simply allowing students to
intermit their studies.
\end{enumerate}

Hence we should ask ourselves \emph{why} such courses are so
successful (if indeed they are: there is a possible confounding
factor, in that, for those universities with scanty support for the
sandwich system, those sudents that do take a sandwich year wil tend
to be the more self-motivated ones, who would probably do well
anyway).  %Thanks to Matt for this!  There are, it seems to the
authors, two classes of reasons: those instrinsic to the sandwich
process, and the skills the sandwich process confers. The first class
is easy to understand: the employer can view the year as a year-long
assessment phase before deciding whether to offer a permanent
job. Bath's experience is that about 2/3 of sandwich placements result
in job offers to the student. However, it is the second class that we
need to investigate. A major one, brought out repeatedly by students
returning from placement, is team working.

\subsection{Team/Group working}

``Teamwork'' is often identified as a key skill \cite[and many
others]{ArcherDavidson2008}. Simplistically, then, universities should
teach it. Indeed, the British Computer Society has long required this
of degrees it accredits:
\begin{quote} An ability to work as a member of a development team
recognising the different roles within a team and different ways of
organising teams \cite[Requirement 2.3.1]{BCS2018a}.
\end{quote}
The problem is that group working is unpopular with students. Most of them have never experienced it in their academic work at school, and really dislike ``being dragged down'', as they generally put it, by others. In many countries, this wouldn't matter, but the UK has (many would say ``suffers from'', see, e.g. \cite{Cupples2015a}) the National Student Survey \cite{OfS2018a}. The results of this, notably the headline ``are you satisfied with your course'' question, are pored over intently by university management \cite[Myth 3]{OfS2018a}. The students submit free-text responses, and it does not take many students complaining about the group work before the Pro-Vice-Chancellor (Teaching) or equivalent is beating down the Director of Studies door, saying ``obviously you must abolish groupwork immediately''. Both the first and third authors have been Directors of Studies, and can vouch for this.


% this is the main section? do we have text to rip from the IoC bid doc?
\section{Institute of Coding}

Formally announced in \cite{DfE2018a}, but foreshadowed in
\cite{HMG2015a}.

% From CfP:
% Short papers (up to 5 pages) focus on dissemination and discussion
% of new ideas in computing education practice or research that merit
% wider awareness and discussion within the community. They can present
% preliminary results of new educational innovations...

% TC: so are we going to talk about the IoC "national" model, with
% industry matched funding, etc?

\section{Anticipated Impact}

\section{Conclusions and Future Work}

% TC: what is the main takehome from this paper? Replicability to
% other jurisdictions? Introducing the IoC as an effective
% educational/industrial collaboerative intervention? Recognising all
% of the mistakes/missed opportunitirs from before, even though
% significant government policy in this space, etc? Global
% competitiveness, etc
% also: directly linking to CAS/BCS work, curriculum+quals reform, pipelines, perceptions?

\section{Acknowledgements}

This work was supported by the Institute of Coding which received
\pounds20m of funding from the Office for Students (OfS), as well as
support from the Higher Education Funding Council for Wales (HEFCW).
The first author is grateful to Matt Dickson (University of Bath) for
discussions about \cite{DfE2018d}, but any mistakes are the authors'
alone.


\bibliographystyle{ACM-Reference-Format}
\bibliography{SIGCSE2019} 

\end{document}
